\documentclass[12pt]{article}
\usepackage[russian]{babel}
\usepackage{amsfonts}
\usepackage{amsmath}
\usepackage{amssymb}
\usepackage{graphicx}
\usepackage{pgf,tikz,pgfplots}
\usepackage{mathrsfs}
\pgfplotsset{compat=1.15}
\usetikzlibrary{arrows}
\pagestyle{empty}
\usepackage{amssymb,fancyhdr,txfonts,pxfonts}
\begin{document}
\pagestyle{plain}
\hspace{4cm} Введение в Фурье-оптику

$R=\frac{\lambda}{(\delta\lambda)min}=N.m$

$d\sin\theta=m\lambda$

$G=\frac{\lambda}{m}=\Delta\lambda$

$p=\frac{\sqrt{z\lambda'}}{D}>>1$

$\frac{\sqrt{z\lambda'}}{D}=Nd$


Почему дифракция решетка монохром свет

\begin{tikzpicture}[line cap=round,line join=round,>=triangle 45,x=1cm,y=1cm]
\clip(-2.688741429757562,-1.8998534990842273) rectangle (3.1138413853721363,1.6896000070491357);
\draw [rotate around={90:(0,0)},line width=1pt] (0,0) ellipse (1.5cm and 0.5cm);
\draw [line width=1pt] (1,-1.8998534990842273) -- (1,1.6896000070491357);
\draw [line width=1pt] (-1.5,-0.7)-- (-0.7,-0.7);
\draw [line width=1pt] (-1.5,0.5)-- (-0.8973064928324016,0.49571510097659144);
\draw [line width=1pt] (-1.4,0.75)-- (-1.4,-0.9);
\draw [line width=1pt,dash pattern=on 1pt off 1pt] (-0.95,1.2)-- (-0.95,-1.45);
\draw [->,line width=1pt] (-0.95,0.5) -- (-0.08000709746922396,1.48067207010565);
\draw [->,line width=1pt] (-0.95,-0.7) -- (-0.49527041606574257,-0.20582743920720592);
\draw [->,line width=1pt] (-0.95,-1.3862017899787233) -- (-0.42704453980101276,-0.7801901365918894);
\draw [line width=1pt] (-0.08000709746922399,1.48067207010565)-- (1,1.5);
\draw [line width=1pt] (-0.49527041606574257,-0.20582743920720595)-- (1,1.5);
\draw [line width=1pt] (-0.42704453980101276,-0.7801901365918894)-- (1,1.5);
\draw (1.1594610254375608,1.601515258432366) node[anchor=north west] {p};
\draw (-1.5216185105853643,0.3573181842204948) node[anchor=north west] {d};
\draw (1.0933974639749835,-1.7181787050621398) node[anchor=north west] {Э};
\end{tikzpicture}


Дельта импульс падает на дифррен . Отставание по времени от щелей $\frac{d\sin\theta}{C}$

\begin{tikzpicture}[line cap=round,line join=round,>=triangle 45,x=1cm,y=1cm]
\clip(-1.7871543464242057,-1.4753980915803246) rectangle (2.242592639412647,1.0173865713851749);
\draw [shift={(-0.8,1)},line width=1pt]  plot[domain=-1.6951513213416582:0,variable=\t]({1*0.8062257748298549*cos(\t r)+0*0.8062257748298549*sin(\t r)},{0*0.8062257748298549*cos(\t r)+1*0.8062257748298549*sin(\t r)});
\draw [shift={(-0.8,1)},line width=1pt]  plot[domain=-1.6951513213416582:0,variable=\t]({1*0.8062257748298549*cos(\t r)+0*0.8062257748298549*sin(\t r)},{0*0.8062257748298549*cos(\t r)+1*0.8062257748298549*sin(\t r)});
\draw [shift={(0.8,1)},line width=1pt]  plot[domain=3.141592653589793:4.8411258934489325,variable=\t]({1*0.7937742251701452*cos(\t r)+0*0.7937742251701452*sin(\t r)},{0*0.7937742251701452*cos(\t r)+1*0.7937742251701452*sin(\t r)});
\draw [line width=1pt] (-0.9801212557351916,0.15865771408380178)-- (1.0552899896969135,0.1633693141889687);
\draw [line width=1pt,dash pattern=on 1pt off 1pt] (1.1500680002923966,1.1436523717937332)-- (1.15,-0.31);
\draw [line width=1pt] (1.153311085704936,0.8374131560748982)-- (1.7259903201542826,1.0334308806179626);
\draw [line width=1pt] (1.16,0.5)-- (1.6,0.5);
\draw [line width=1pt] (1.150038514063448,0.5133208667992908)-- (1.5613208733533368,0.6593732876285148);
\draw (1.3861758226883065,0.5853548736626266) node[anchor=north west] {$\theta$   };
\draw [line width=1pt] (-0.8684238700028556,-1.549311383567656)-- (-0.8684238700028556,-1.549311383567656)-- (-0.8401037372274982,-1.5587514278261083)-- (-0.8117836044521407,-1.5115512065338461)-- (-0.7834634716767833,-1.4737910295000363)-- (-0.7740234274183309,-1.4454708967246788)-- (-0.7551433389014259,-1.4171507639493213)-- (-0.7457032946429735,-1.369950542657059)-- (-0.7268232061260685,-1.3321903656232492)-- (-0.7173831618676161,-1.3038702328478917)-- (-0.7173831618676161,-0.9923487723189605)-- (-0.7173831618676161,-0.9262684625097931);
\draw [line width=1pt] (-0.7551433389014259,-0.964028639543603)-- (-0.7551433389014259,-0.964028639543603)-- (-0.7645833831598784,-0.9923487723189605)-- (-0.7551433389014259,-1.0395489936112228)-- (-0.736263250384521,-1.086749214903485)-- (-0.7268232061260685,-1.1150693476788425)-- (-0.7079431176091636,-1.1433894804542)-- (-0.6890630290922586,-1.1717096132295572)-- (-0.6701829405753537,-1.209469790263367)-- (-0.6418628077999963,-1.2566700115556295)-- (-0.6324227635415438,-1.2849901443309868)-- (-0.6041026307661864,-1.3133102771063443)-- (-0.594662586507734,-1.3416304098817016)-- (-0.5663424537323765,-1.369950542657059)-- (-0.5474623652154715,-1.3982706754324163)-- (-0.5191422324401141,-1.4171507639493213)-- (-0.4908220996647567,-1.4265908082077738)-- (-0.4625019668893993,-1.4360308524662264)-- (-0.4247417898555894,-1.4360308524662264)-- (-0.396421657080232,-1.4265908082077738)-- (-0.3681015243048746,-1.4171507639493213)-- (-0.33034134727106473,-1.388830631173964)-- (-0.28314112597880237,-1.3605104983986065)-- (-0.28314112597880237,-1.162269568971105)-- (-0.28314112597880237,-1.0961892591619375)-- (-0.29258117023725483,-1.06786912638658)-- (-0.29258117023725483,-0.8696281969590783)-- (-0.29258117023725483,-0.8035478871499111);
\draw [line width=1pt] (-0.33034134727106473,-0.8507481084421734)-- (-0.33034134727106473,-0.8507481084421734)-- (-0.33034134727106473,-0.8885082854759833)-- (-0.32090130301261227,-0.9168284182513406)-- (-0.31146125875415975,-0.945148551026698)-- (-0.29258117023725483,-0.9734686838020555)-- (-0.2737010817203499,-1.001788816577413)-- (-0.2453809489449925,-1.0206689050943178)-- (-0.21706081616963507,-1.0489890378696751)-- (-0.19818072765273012,-1.0773091706450326)-- (-0.1698605948773727,-1.10562930342039)-- (-0.16042055061892024,-1.1339494361957474)-- (-0.13210041784356283,-1.1528295247126523)-- (-0.12266037358511035,-1.1811496574880098)-- (-0.10378028506820541,-1.209469790263367)-- (-0.08490019655130047,-1.2377899230387246)-- (-0.05658006377594306,-1.2661100558140819)-- (-0.028259931000585648,-1.2755501000725344)-- (0.009500246033224236,-1.2755501000725344)-- (0.047260423067034124,-1.2755501000725344)-- (0.09446064435929648,-1.2566700115556295)-- (0.14166086565155883,-1.247229967297177)-- (0.16998099842691625,-1.2377899230387246)-- (0.20774117546072612,-1.2377899230387246)-- (0.23606130823608354,-1.2189098345218197)-- (0.26438144101144095,-1.2000297460049147)-- (0.2832615295283459,-1.1717096132295572)-- (0.2927015737867984,-1.124509391937295)-- (0.3115816623037033,-1.086749214903485)-- (0.3115816623037033,-0.7374675773407439)-- (0.3115816623037033,-0.6713872675315766);
\draw [line width=1pt] (0.2738214852698934,-0.699707400306934)-- (0.2738214852698934,-0.699707400306934)-- (0.2832615295283459,-0.7374675773407439)-- (0.2927015737867984,-0.7657877101161013)-- (0.36822192785441815,-0.8790682412175308)-- (0.405982104888228,-0.9357085067682456)-- (0.424862193405133,-0.9734686838020555)-- (0.45318232618049037,-1.0206689050943178)-- (0.46262237043894283,-1.0489890378696751)-- (0.4720624146973953,-1.0773091706450326)-- (0.4909425032143003,-1.10562930342039)-- (0.5287026802481102,-1.10562930342039)-- (0.56646285728192,-1.10562930342039)-- (0.5947829900572774,-1.0961892591619375)-- (0.6231031228326348,-1.086749214903485)-- (0.6514232556079923,-1.06786912638658)-- (0.6891834326418022,-1.0489890378696751)-- (0.7363836539340645,-1.0395489936112228)-- (0.7741438309678744,-1.0206689050943178)-- (0.8213440522601367,-1.0112288608358653)-- (0.8213440522601367,-0.9262684625097931)-- (0.8213440522601367,-0.8601881527006259)-- (0.8119040080016843,-0.8318680199252685)-- (0.7930239194847793,-0.8035478871499111)-- (0.7741438309678744,-0.7563476658576488)-- (0.7552637424509695,-0.699707400306934)-- (0.745823698192517,-0.661947223273124)-- (0.745823698192517,-0.539226647913242)-- (0.745823698192517,-0.4731463381040748);
\draw [line width=1pt] (0.7363836539340645,-0.5203465593963371)-- (0.7363836539340645,-0.5203465593963371)-- (0.726943609675612,-0.5486666921716945)-- (0.726943609675612,-0.5864268692055044)-- (0.7363836539340645,-0.6147470019808617)-- (0.7552637424509695,-0.6430671347562191)-- (0.764703786709422,-0.6713872675315766)-- (0.7930239194847793,-0.699707400306934)-- (0.8024639637432318,-0.7280275330822914)-- (0.8213440522601367,-0.7563476658576488)-- (0.8402241407770417,-0.7846677986330062)-- (0.906304450586209,-0.8979483297344357)-- (0.9440646276200189,-0.9545885952851505)-- (0.9723847603953762,-0.982908728060508)-- (1.0007048931707336,-1.001788816577413)-- (1.0290250259460911,-0.9923487723189605)-- (1.0290250259460911,-0.5297866036547896)-- (1.0290250259460911,-0.46370629384562234)-- (1.0195849816876386,-0.4353861610702649)-- (1.0195849816876386,-0.39762598403645505)-- (1.0195849816876386,-0.4353861610702649)-- (1.0195849816876386,-0.4731463381040748)-- (1.0290250259460911,-0.5014664708794322)-- (1.0384650702045435,-0.5297866036547896)-- (1.0384650702045435,-0.5675467806885994)-- (1.0384650702045435,-0.6053069577224093)-- (1.047905114462996,-0.6430671347562191)-- (1.066785202979901,-0.6902673560484816)-- (1.0762252472383536,-0.7185874888238389)-- (1.1045453800137108,-0.7374675773407439)-- (1.1328655127890683,-0.7752277543745537)-- (1.1611856455644256,-0.822427975666816)-- (1.189505778339783,-0.8507481084421734)-- (1.2178259111151406,-0.8696281969590783)-- (1.2461460438904979,-0.8885082854759833)-- (1.2744661766658554,-0.8979483297344357)-- (1.3122263536996652,-0.8979483297344357)-- (1.3405464864750227,-0.8885082854759833)-- (1.349986530733475,-0.841308064183721)-- (1.36886661925038,-0.8035478871499111);
\draw [->,line width=1pt] (-0.8212236487105932,-1.634271781893728) -- (1.5104672831271673,-0.8979483297344357);
\end{tikzpicture}

$\Delta t=\frac{Nd\sin\theta}{C}$ :Длительность цуга 

$\Delta t\Delta f\approx1$

$\frac{f}{\Delta f}=\frac{C}{Nm\lambda}=\Delta f=\frac{C}{Nd\sin\theta}$

$\frac{f}{\Delta f}=\frac{\lambda}{\Delta\lambda}=N.m$




\newpage \hspace{5cm} Призма

Дисперсия- зависимость коэффициента преломления от длины волны или частоты.Чем больше длины волны, тем меньше n.  Спектр за призмой мы не выходе призмы очень широкие.

Как получить спектр?

1.	Отойти подольше;

2.       Поставить линзу.

Почему во 2м случае изображение будут иметь конечный позмер. 

\begin{tikzpicture}[line cap=round,line join=round,>=triangle 45,x=1cm,y=1cm]
\clip(-2.0569706072808915,-1.2819302401497359) rectangle (4.545305965057389,2.802210485812848);
\draw [line width=1pt] (0,1)-- (-0.5,0);
\draw [line width=1pt] (-0.5,0)-- (0.5,0);
\draw [line width=1pt] (0,1)-- (0.5,0);
\draw [rotate around={90:(1.2,0.2)},line width=1pt] (1.2,0.2) ellipse (0.9486832980505155cm and 0.3cm);
\draw [line width=1pt,domain=-2.0569706072808915:0] plot(\x,{(-1-0.5*\x)/-1});
\draw [line width=1pt,dash pattern=on 1pt off 1pt,domain=0:4.545305965057389] plot(\x,{(--0.8228681699412685--0.4067356681492713*\x)/0.8228681699412685});
\draw [line width=1pt] (-0.4140458955898584,0.17190820882028324)-- (-1.4335538038438271,-0.3861467772483843);
\draw [line width=1pt] (0,1)-- (1.3767925181402483,0.9664489906682007);
\draw [line width=1pt] (0,1)-- (0.962645465243806,0.780196732416436);
\draw [line width=1pt] (2.5749226889325216,1.0953682444890382)-- (1.7419199031254422,-1.2191495630554634);
\draw [line width=1pt] (1.3767925181402483,0.9664489906682007)-- (2.3088696960927093,0.35613375431667427);
\draw [line width=1pt] (0.962645465243806,0.780196732416436)-- (2.212466201121157,0.08827437902658941);
\draw [line width=1pt] (0.5,0)-- (1.2036447867195383,-0.6476942450005657);
\draw [line width=1pt] (1.2036447867195383,-0.6476942450005657)-- (2.303442272606557,0.35562977721215344);
\draw [line width=1pt] (0.5,0)-- (1.6474227196213176,-0.5640839098161725);
\draw [line width=1pt] (1.4178329253387654,-0.45229453959355137)-- (2.212466201121157,0.08827437902658941);
\draw [line width=1pt] (1.3767925181402483,0.9664489906682007)-- (2.3088696960927093,0.35613375431667427);
\draw [line width=1pt] (1.4178329253387654,-0.45229453959355137)-- (1.6474227196213176,-0.5640839098161725);
\draw [line width=1pt] (2.303442272606557,0.35562977721215344)-- (1.2036447867195383,-0.6476942450005657);
\draw [line width=1pt] (0.962645465243806,0.780196732416436)-- (0,1);
\draw [->,line width=0.5pt] (0.058826351117846454,-0.24250569756850604) -- (0.48330959128476586,-0.23607413332355273);
\draw [->,line width=0.5pt] (-0.1276890119858,-0.24250569756850607) -- (-0.47499348121327944,-0.24250569756850607);
\draw (-0.2,0.7) node[anchor=north west] {n($\lambda$)};
\draw (0.9,1.8) node[anchor=north west] { $\phi$($\lambda$)};
\draw (-2,0.5) node[anchor=north west] {$\lambda$,$\lambda$+$\delta$$\lambda$};
\draw (-0.1,-0.2) node[anchor=north west] {b};
\draw [->,line width=0.5pt] (0.5862146192040191,-0.5) -- (0.8240810242572654,-0.1593213103890781);
\draw (-0.2,-0.5) node[anchor=north west] {D-ширина пучка};
\end{tikzpicture}

Ответ: Каждый из щитов имеет конечный размер следовательно он имеет дифракционную расходимость А/D- это угловая полуширина пучка который имеет ширину D и длину вольны А.   

$d\phi=(\frac{d\phi}{d\lambda})\delta\phi \geq \frac{\lambda}{D} \Rightarrow$  То не увидим 2 пучка

 $R=\frac{\lambda}{(\delta\lambda)min}=D(\frac{d\phi}{d\lambda}) \Rightarrow$ b$\left|\frac{dn}{d\lambda}\right|$ 

Пример: $\frac{dn}{d\lambda}<0$

Для хороших стекол $\left|\frac{dn}{d\lambda}\right|$ $\approx 10^3$ см$^-1$ 

Ex. 589.0 нм

589.6 нм

$\Delta\lambda=0.6$ нм

Какое нужно разрешение?

$R=\frac{\lambda}{\Delta\lambda}=$$10^3$

Ex. n$\approx 1200$ 1/мм

m=2, L=100cм, R$\approx 2.10^5$
\newpage \hspace{5cm} Интерферометр Фабри-Перо

Простейший пример: 2 плоено параллельных зерня с как можно большим коэффициентом отражения ( отражательной способностью).

(1-r)- коэффициент 

r- энергетический коэффициент отражений

Задача сложить интенсивность всех пучков, фазы которых составляют арифметическую процессию. И сосчитать R-?

\begin{tikzpicture}[line cap=round,line join=round,>=triangle 45,x=1cm,y=1cm]
\clip(-2.9745243336323006,-2.5) rectangle (9.341866671553333,2);
\draw [line width=2pt] (1,1.5)-- (1,-1.5);
\draw [line width=2pt] (-1,1.5)-- (-1,-1.5);
\draw [->,line width=2pt] (-2.5,1) -- (-1,1);
\draw [->,line width=2pt] (1,1) -- (2.8,1);
\draw [line width=2pt] (-1,1)-- (1,1);
\draw [line width=2pt] (1,1)-- (-1,0.7201430195973559);
\draw [line width=2pt] (-1,0.7201430195973559)-- (1,0.36135397577719064);
\draw [line width=2pt] (-1,0)-- (1,-0.46952591517477105);
\draw [line width=2pt] (1,-0.46952591517477105)-- (-1,-1);
\draw [line width=2pt] (1,0.36135397577719064)-- (-1,0);
\draw [line width=2pt] (1,-1.3192894400120045)-- (-1,-1);
\draw [->,line width=2pt] (0.9337554167195674,0.3732378911239165) -- (2.8,0.37);
\draw [->,line width=2pt] (1,-0.46952591517477105) -- (2.8,-0.47);
\draw [->,line width=2pt] (-1,0.7201430195973559) -- (-2.0229441511618433,0.7201430195973559);
\draw (2.985019701134942,1.3378717944318848) node[anchor=north west] {1};
\draw (3.242098071262156,0.4731536403676196) node[anchor=north west] {2};
\draw (3.1836711689605166,-0.344822991855334) node[anchor=north west] {3};
\draw [line width=2pt] (1.2250408771049202,1.0222811617617056)-- (1,1.2677684022702398);
\draw [line width=2pt] (1.1872736093343763,0.3991212435477344)-- (1,0.6068412162857246);
\draw [line width=2pt] (1.3194590465312794,-0.5072931829453148)-- (1,-0.18627140689569321);
\draw [line width=2pt] (1,-0.8471985928802082)-- (1.2250408771049202,-1.0926858333887424);
\draw [line width=2pt] (-1,1.31917518569536)-- (-1.136413987222718,1.1263664080999496);
\draw [line width=2pt] (-1,1)-- (-1.1492679057290784,0.805018445440932);
\draw [line width=2pt] (-1,0.5993557493391608)-- (-1.1621218242354392,0.3808391347310289);
\draw [line width=2pt] (-1,0.25229994966742186)-- (-1.1621218242354392,0.05949117207201139);
\draw [line width=2pt] (-1,-0.2104411165615634)-- (-1.1621218242354392,-0.3903959756506131);
\draw [line width=2pt] (-1,-0.5446429977269416)-- (-1.1492679057290784,-0.7245978568159913);
\draw [line width=2pt] (-1,-0.8659909603859592)-- (-1.1492679057290784,-1.0459458194750089);
\draw [line width=2pt] (-1,-1.1873389230449765)-- (-1.1749757427418,-1.4058555376531086);
\draw [line width=2pt] (1,0.1751764386292577)-- (1.1644374254158507,0);
\draw [line width=2pt] (1,-0.5574969162333022)-- (1.1901452624285722,-0.7245978568159913);
\draw [line width=2pt] (1,-1.1616310860322552)-- (1.1644374254158507,-1.2901702710958622);
\draw (1.0218757837998504,1.9221408174482801) node[anchor=north west] {r};
\draw (-1.1048634599798324,1.8870846760672966) node[anchor=north west] {r};
\draw [->,line width=2pt] (0.21558453203722336,-1.8756078321582899) -- (1,-1.88);
\draw [->,line width=2pt] (-0.22,-1.88) -- (-0.94,-1.88);
\draw (-0.2,-1.735383266634355) node[anchor=north west] {L};
\end{tikzpicture}

$\Delta=2L$

$m=\frac{2L}{\lambda}$- очень большое число

Пугни тиспокешреально сподшотю интенсивности и амплитуде. Примем, что интенсивность (1-r) и остальные пути не теряют в интенсивности, тогда конечная интенсивность.

$\left\{ \begin{array}{l} \frac{1}{1-r}=N \\  R=m.\frac{1}{1-r}=\frac{2L}{\lambda}.\frac{1}{1-r} \end{array} \right.$

Обычные бытовые зеркала покрывало амальгамой $r\approx 0.9-0.92$ Лучше серебро, но ненадолго $r\approx 0.96$ Еще лучше- множество слоев диэлектрика, $r\approx 0.999$, $\Delta$-мало.

Ex. L=100 cм

$\lambda=5.10^-5$ см

r=0.9

m=$4.10^5$, G=$\frac{\lambda}{m}=\frac{500}{4.10^5}\approx 10^-3$ нм

\begin{tikzpicture}[line cap=round,line join=round,>=triangle 45,x=1cm,y=1cm]
\clip(-0.2,-0.5) rectangle (4.8,2.1);
\draw [->,line width=1pt] (0,0) -- (3,0);
\draw [->,line width=1pt] (0,0) -- (0,2);
\draw [line width=1pt,dash pattern=on 1pt off 1pt,domain=0:5.076815783831555] plot(\x,{(--1.74-0*\x)/1.74});
\draw [line width=1pt] (0.06629257358663879,0.061099854326487124)-- (0.06629257358663879,0.061099854326487124)-- (0.08842846031609494,0.06847848323630583)-- (0.12532160486518856,0.08323574105594325)-- (0.16221474941428218,0.09799299887558066)-- (0.19172926505355709,0.12012888560503679)-- (0.2286224096026507,0.1422647723344929)-- (0.2507582963321069,0.16440065906394905)-- (0.27289418306156304,0.17915791688358645)-- (0.2876514408812005,0.20129380361304258)-- (0.302408698700838,0.2234296903424987)-- (0.3171659565204754,0.26032283489159225)-- (0.3393018432499315,0.289837350530867)-- (0.3466804721597503,0.3341091239897793)-- (0.3614377299793877,0.3636236396290541)-- (0.3688163588892065,0.3857595263585103)-- (0.37619498779902516,0.4078954130879664)-- (0.37619498779902516,0.43740992872724116)-- (0.3909522456186626,0.45954581545669737)-- (0.39833087452848137,0.4816817021861534)-- (0.40570950343830003,0.5111962178254281)-- (0.40570950343830003,0.540710733464703)-- (0.4130881323481188,0.5628466201941593)-- (0.4130881323481188,0.6071183936530715)-- (0.4130881323481188,0.6513901671119837)-- (0.42046676125793747,0.6735260538414398)-- (0.42784539016775625,0.7030405694807147)-- (0.42784539016775625,0.8358558898574515)-- (0.42784539016775625,0.9022635500458198)-- (0.435224019077575,0.9391566945949132)-- (0.4573599058070311,0.9539139524145508);
\draw [line width=1pt] (0.46248511680832094,0.9282170256908855)-- (0.46248511680832094,0.9282170256908855)-- (0.48462100353777715,0.913459767871248)-- (0.48462100353777715,0.8839452522319734)-- (0.48462100353777715,0.8544307365926985)-- (0.4919996324475958,0.8322948498632424)-- (0.5067568902672334,0.8101589631337863)-- (0.514135519177052,0.7806444474945115)-- (0.5288927769966895,0.7437513029454178)-- (0.5584072926359643,0.6994795294865056)-- (0.5879218082752392,0.6552077560275934)-- (0.595300437185058,0.6330718692981373)-- (0.6100576950046954,0.6109359825686812)-- (0.6174363239145142,0.5888000958392249)-- (0.6248149528243329,0.5666642091097688)-- (0.6469508395537891,0.5445283223803127)-- (0.6690867262832452,0.5150138067410378)-- (0.6912226130127014,0.47812066219194427)-- (0.6986012419225202,0.4486061465526694)-- (0.7133584997421576,0.41171300200357597)-- (0.7207371286519763,0.38957711527411987)-- (0.7207371286519763,0.35268397072502633)-- (0.7207371286519763,0.3231694550857515)-- (0.728115757561795,0.29365493944647664)-- (0.7428730153814325,0.2567617948973831)-- (0.7650089021108887,0.24200453707774566)-- (0.7871447888403448,0.22724727925810828)-- (0.809280675569801,0.21986865034828956)-- (0.8756883357581695,0.19035413470901474)-- (0.9347173670367193,0.1608396190697399)-- (0.9568532537661755,0.1534609901599212)-- (0.9863677694054506,0.1534609901599212)-- (1.0085036561349066,0.1460823612501025)-- (1.0380181717741812,0.1460823612501025)-- (1.0675326874134563,0.1460823612501025)-- (1.0896685741429124,0.1534609901599212)-- (1.1118044608723685,0.1608396190697399)-- (1.1339403476018246,0.17559687688937733)-- (1.1486976054214622,0.19773276361883346)-- (1.1634548632410997,0.21986865034828956)-- (1.1708334921509183,0.2567617948973831)-- (1.1855907499705558,0.286276310536658)-- (1.1929693788803748,0.3084121972661141)-- (1.1929693788803748,0.3379267129053889)-- (1.2003480077901933,0.37481985745448243)-- (1.215105265609831,0.41171300200357597)-- (1.229862523429468,0.4707420332821257)-- (1.2446197812491056,0.5150138067410378)-- (1.2446197812491056,0.5445283223803127)-- (1.2446197812491056,0.5740428380195876)-- (1.2519984101589245,0.6183146114784998)-- (1.2667556679785617,0.6478291271177746)-- (1.2741342968883806,0.6921009005766868)-- (1.2888915547080182,0.7216154162159617)-- (1.2888915547080182,0.7511299318552366)-- (1.2962701836178367,0.78802307640433)-- (1.3110274414374743,0.8175375920436049)-- (1.3184060703472928,0.8544307365926985)-- (1.3331633281669304,0.8839452522319734)-- (1.3331633281669304,0.913459767871248)-- (1.347920585986568,0.9355956546007045)-- (1.2888915547080182,0.8839452522319734)-- (1.2888915547080182,0.8839452522319734)-- (1.3331633281669304,0.8839452522319734)-- (1.3774351016258426,0.8987025100516107)-- (1.4143282461749362,0.9060811389614295)-- (1.4364641329043923,0.9208383967810668)-- (1.4659786485436674,0.9208383967810668)-- (1.480735906363305,0.8987025100516107)-- (1.480735906363305,0.8691879944123359)-- (1.4881145352731235,0.8470521076828798)-- (1.4881145352731235,0.8175375920436049)-- (1.495493164182942,0.7954017053141488)-- (1.502871793092761,0.7732658185846927)-- (1.502871793092761,0.7216154162159617)-- (1.502871793092761,0.6773436427570495)-- (1.5102504220025796,0.6478291271177746)-- (1.5250076798222172,0.6035573536588624)-- (1.5250076798222172,0.5740428380195876)-- (1.5323863087320357,0.5445283223803127)-- (1.5471435665516733,0.5002565489214004)-- (1.5619008243713108,0.47812066219194427)-- (1.5692794532811294,0.45598477546248817)-- (1.5914153400105855,0.44122751764285084)-- (1.5987939689204045,0.4190916309133947)-- (1.613551226740042,0.3969557441839386)-- (1.6283084845596791,0.37481985745448243)-- (1.6504443712891357,0.35268397072502633)-- (1.6799588869284103,0.3157908261759328)-- (1.716852031477504,0.27889768162683926)-- (1.73898791820696,0.2641404238072018)-- (1.7611238049364162,0.2567617948973831)-- (1.7832596916658727,0.24200453707774566)-- (1.7980169494855098,0.21986865034828956)-- (1.820152836214966,0.2124900214384709)-- (1.8422887229444225,0.19773276361883346)-- (1.8718032385836971,0.19035413470901474)-- (1.9086963831327908,0.17559687688937733)-- (1.9308322698622469,0.16821824797955862)-- (1.960346785501522,0.16821824797955862)-- (1.9898613011407966,0.16821824797955862)-- (2.0119971878702536,0.17559687688937733)-- (2.03413307459971,0.19773276361883346)-- (2.0415117035095287,0.21986865034828956)-- (2.0636475902389853,0.24200453707774566)-- (2.071026219148804,0.2641404238072018)-- (2.085783476968442,0.286276310536658)-- (2.0931621058782603,0.3231694550857515)-- (2.1079193636978975,0.35268397072502633)-- (2.115297992607716,0.38957711527411987)-- (2.130055250427354,0.4190916309133947)-- (2.1374338793371725,0.44122751764285084)-- (2.1374338793371725,0.4707420332821257)-- (2.144812508246991,0.4928779200115817)-- (2.1521911371568097,0.5150138067410378)-- (2.1521911371568097,0.5445283223803127)-- (2.159569766066628,0.5814214669294063)-- (2.174327023886266,0.6109359825686812)-- (2.1817056527960847,0.6478291271177746)-- (2.196462910615722,0.6773436427570495)-- (2.2038415395255413,0.6994795294865056)-- (2.2038415395255413,0.8544307365926985)-- (2.2038415395255413,0.9060811389614295)-- (2.21122016843536,0.9282170256908855)-- (2.21122016843536,0.9577315413301606);
\draw [line width=1pt] (2.2134735863438877,0.9391566945949134)-- (2.2134735863438877,0.9391566945949134)-- (2.2134735863438877,0.9096421789556385)-- (2.2208522152537062,0.8801276633163637)-- (2.228230844163525,0.8579917765869075)-- (2.228230844163525,0.8284772609476327)-- (2.228230844163525,0.7989627453083579)-- (2.2429881019831623,0.7473123429396269)-- (2.2577453598028,0.7030405694807147)-- (2.2577453598028,0.6735260538414398)-- (2.2651239887126184,0.6513901671119837)-- (2.2651239887126184,0.47430307327633475)-- (2.2651239887126184,0.40789541308796634)-- (2.2725026176224374,0.38575952635851024)-- (2.2725026176224374,0.3562450107192354)-- (2.279881246532256,0.33410912398977927)-- (2.2946385043518935,0.31197323726032317)-- (2.302017133261712,0.289837350530867)-- (2.3167743910813496,0.2677014638014109)-- (2.3389102778108057,0.2529442059817735)-- (2.3462889067206247,0.23080831925231737)-- (2.3758034223598994,0.20867243252286125)-- (2.3979393090893555,0.20129380361304253)-- (2.54551188728573,0.20129380361304253)-- (2.60454091856428,0.20129380361304253)-- (2.6930844654821047,0.1939151747032238)-- (2.7521134967606544,0.1939151747032238);
\draw [line width=1pt,dash pattern=on 1pt off 1pt] (0.49786047695426405,0.8235035831032405)-- (0.5,0);
\draw [line width=1pt,dash pattern=on 1pt off 1pt] (1.3694635463770546,0.8960453249686776)-- (1.402739112327029,0);
\draw [line width=1pt,dash pattern=on 1pt off 1pt] (2.192435107067253,0.6692880356601194)-- (2.236524179136545,0);
\draw (2.9303055555206523,-0.10207422766749469) node[anchor=north west] { $\lambda$};
\draw (-0.4,1.0565534523866815) node[anchor=north west] {1};
\draw (0.12520696170526843,2.0444360006434) node[anchor=north west] {$I_0$};
\end{tikzpicture}

 Интерферометр Ф-П необходимо поспать пучок шириной более а, т.е необходимо предварительное сужение пучка| $R=.10^7$.

\begin{tabular}{c|c}
Прибор призма  & $\partial 0^R 10^4$ \\ \hline
Реш. ка & $\partial0$  $10^6$ \\ \hline
Ф-П & $\partial0$  $10^8-10^9$ \\ \hline
\end{tabular}

    Рассмотрим Ф-п как оптический резонатор и кашлем на него очень узкий импульс. Помарается резонатор резонирующий на многих частотах.

\begin{tikzpicture}[line cap=round,line join=round,>=triangle 45,x=1cm,y=1cm]
\clip(-5.1784738163013255,-1.3700698140921848) rectangle (3.209984187780153,2.5);
\draw [line width=1pt] (-5,0)-- (-4.1917670295059635,0);
\draw [line width=1pt] (-4.212858329202477,0)-- (-4.02,1.8);
\draw [line width=1pt] (-4.02,1.8)-- (-3.867365817392096,0);
\draw [line width=1pt] (-3.867365817392096,0)-- (-3.1413250093277263,0);
\draw [line width=1pt] (-2.4152842012633564,1.858598790474038)-- (-2.4152842012633564,0);
\draw [line width=1pt] (-0.82,1.86)-- (-0.8241734942286735,0);
\draw[line width=1pt,smooth,samples=80,domain=0:2] plot(\x,{(\x)^-0.5});
\draw [->,line width=0.5pt] (0,0) -- (0,2);
\draw [->,line width=0.5pt] (0,0) -- (2.7332573068427966,0);
\draw [line width=1pt] (0.6012225794513826,1.2896811693101826)-- (0.6,0);
\draw [line width=1pt] (1.1884896301100953,0.9172807966834436)-- (1.19,0);
\draw [line width=1pt] (1.7119233056972092,0.7642894201834359)-- (1.71,0);
\draw [->,line width=0.5pt] (-1.5435788229543514,-0.2957765968584276) -- (-0.7903449971094806,-0.2957765968584276);
\draw [->,line width=0.5pt] (-1.89,-0.3) -- (-2.42,-0.3);
\draw (-1.7951872200605632,-0.24521421895028833) node[anchor=north west] {L};
\draw (2.73039459295221,-0.17545573242986065) node[anchor=north west] {t};
\end{tikzpicture}

Потеря Интенсивность( энергии) 2(1-r) за время $\frac{2L}{c}$,  $T=\frac{\lambda}{C}$ -период.

$R=\frac{\lambda}{(\delta\lambda)min}=\frac{f}{\Delta f}$ - Q( добротность)

$Q=2\pi$ (заная энергия)/ (Потеря за Т)

Относительная потеря за период:                    
 
2(1-r) $\frac{2L}{C}$\hspace{3cm} $X=\frac{(1-r)\lambda}{L}$

\hspace{3cm} $\Rightarrow$

x $\frac{\lambda}{C}$  \hspace{4cm}$Q=R=\frac{2\pi L}{\lambda (1-r)}\sqrt{2}$

На длине резонатора должно укладываться целое число полувалы. $L=m\frac{\lambda}{2}$
\newpage \hspace{4cm} Введение в Фурье- оптику

\hspace{2cm} Разложение волнового поля по плоским волнам 

Мы предполагаем свет идеально монохроматическими когерентным.

Есть координатная система. В плоскости t=0 создало монохроматическое поле т.е задано распределение номиленный амплитуды как функцимых f(x)- принцип Р–Ф( разложение по соререп. Волносм).

\definecolor{xdxdff}{rgb}{0.49019607843137253,0.49019607843137253,1}
\begin{tikzpicture}[line cap=round,line join=round,>=triangle 45,x=1cm,y=1cm]
\clip(-2,-1.5) rectangle (6.767738589710458,3);
\draw [->,line width=1pt] (0,-1) -- (0,2);
\draw [->,line width=1pt] (0,0) -- (2,0);
\draw [shift={(0,1.1865089520664593)},line width=1pt]  plot[domain=-0.9061157450124702:1.2326976170089197,variable=\t]({1*0.14977474502400345*cos(\t r)+0*0.14977474502400345*sin(\t r)},{0*0.14977474502400345*cos(\t r)+1*0.14977474502400345*sin(\t r)});
\draw [shift={(0.10962993032454832,1.19752492624996)},line width=1pt]  plot[domain=-1.4056476493802688:1.7577755967138085,variable=\t]({1*0.2024938771538399*cos(\t r)+0*0.2024938771538399*sin(\t r)},{0*0.2024938771538399*cos(\t r)+1*0.2024938771538399*sin(\t r)});
\draw [shift={(0.21765700720542241,1.2117903587715153)},line width=1pt]  plot[domain=-1.1902899496825308:1.258754205232362,variable=\t]({1*0.2409376285653408*cos(\t r)+0*0.2409376285653408*sin(\t r)},{0*0.2409376285653408*cos(\t r)+1*0.2409376285653408*sin(\t r)});
\draw [shift={(0,0.77332861349454)},line width=1pt]  plot[domain=-0.6438331618840021:0.7286888708448401,variable=\t]({1*0.14907065710957743*cos(\t r)+0*0.14907065710957743*sin(\t r)},{0*0.14907065710957743*cos(\t r)+1*0.14907065710957743*sin(\t r)});
\draw [shift={(0.1383243534445702,0.7177549321766461)},line width=1pt]  plot[domain=-1.3408844424587905:1.592406361965627,variable=\t]({1*0.19105501865643923*cos(\t r)+0*0.19105501865643923*sin(\t r)},{0*0.19105501865643923*cos(\t r)+1*0.19105501865643923*sin(\t r)});
\draw [shift={(0.21765700720542247,0.7643804146113369)},line width=1pt]  plot[domain=-1.2120256565243235:1.0768549578753144,variable=\t]({1*0.22936033431581232*cos(\t r)+0*0.22936033431581232*sin(\t r)},{0*0.22936033431581232*cos(\t r)+1*0.22936033431581232*sin(\t r)});
\draw (-0.18501194253873907,2.5) node[anchor=north west] {x};
\draw (1.1572178899418009,1.7665786895301376) node[anchor=north west] {P(x,y)};
\draw (2.2757427503422507,0.2) node[anchor=north west] {z};
\draw (0.17,-0.2020250647746462) node[anchor=north west] {z=0};
\draw (1.515145845269945,-0.12149127482581415) node[anchor=north west] {z>0};
\begin{scriptsize}
\draw [fill=xdxdff] (0,1.1865089520664593) circle (1.5pt);
\draw [fill=xdxdff] (0,0.77332861349454) circle (1.5pt);
\end{scriptsize}
\end{tikzpicture}

f(x) по плоскости вольном называется метод Сэлея.
У этих волн одинаковый волновой вектор К, по направление волновых фронтов различны.  

\begin{tikzpicture}[line cap=round,line join=round,>=triangle 45,x=1cm,y=1cm]
\clip(-2,-1.4) rectangle (11.042947397078155,3);
\draw [->,line width=1pt] (0,-0.5430599291908617) -- (0,2);
\draw [->,line width=1pt] (0,0) -- (2.2212801994180684,0);
\draw [line width=1pt] (0,-0.20435717465591952)-- (0.9,-0.2);
\draw [line width=1pt] (0,0.21132347863696388)-- (1.020424978794182,0.21132347863696388);
\draw [line width=1pt] (0,0.503839493917141)-- (1.25,0.5);
\draw [line width=1pt] (1.066611718048947,0.3344881166496701)-- (0.8664691812782991,-0.41989529117815544);
\draw [line width=1pt] (0.8819037295235568,0.5011306427131332)-- (0.6355354850044749,-0.41989529117815544);
\draw [line width=1pt] (0.6201813430178833,0.5019345479219078)-- (0.3892062089790623,-0.41989529117815544);
\draw [line width=1pt] (0.004316614015538047,-0.2043362766101552)-- (1.2975454143227712,0.6577952914330238);
\draw [line width=1pt] (0,0.21132347863696388)-- (1.1435896168068882,1.011893625719554);
\draw [->,line width=1pt] (0,0) -- (1.7132260676156548,1.2120361624902019);
\draw [->,line width=1pt] (0.9,-0.2) -- (1.2051719358132413,-0.6662245672035678);
\draw (-0.27280372033923406,2.3513090641077343) node[anchor=north west] {x};
\draw (2.729334331220482,0.28830137739490524) node[anchor=north west] {z};
\draw (2.005742082895832,2.397495803362499) node[anchor=north west] {P(x,z)};
\draw (1.5900614296029485,1.9) node[anchor=north west] {$\vec K_1$};
\draw (1.2,-0.3) node[anchor=north west] {$\vec K_2$};
\end{tikzpicture}

$E=ae^{-i(\omega t- ke)}$ - плоская волна. 

Коллежская амплитуда: $A=ae^{-i(k\sin \alpha x+k\cos\alpha x)}$ с разложением $\vec K \vec r$ по x,z.

\begin{tikzpicture}[line cap=round,line join=round,>=triangle 45,x=1cm,y=1cm]
\clip(-2,-1.5) rectangle (8,3);
\draw [->,line width=1pt] (0,0) -- (2.2595001290053087,0);
\draw [->,line width=1pt] (0,0) -- (0,2.3100700521604565);
\draw [->,line width=1pt] (0,0) -- (0.34894981149846516,1.4687267930748746);
\draw [->,line width=1pt] (0,0) -- (1.6986046229482537,1.310974931996328);
\draw [line width=1pt] (0.34894981149846516,1.4687267930748746)-- (1.435684854484009,0.39951973465361446);
\draw [line width=1pt] (0.13861399672706953,1.2408629937391962)-- (1.0851251631983498,0.2768238427036338);
\draw (-0.3171136019442876,2.7833256353960962) node[anchor=north west] {x};
\draw (2.6626437739838167,-0.09126383314630822) node[anchor=north west] {z};
\draw (-0.4047535247657024,0.06648802793223836) node[anchor=north west] {0};
\draw (0.2788378732413333,2.2399581139033247) node[anchor=north west] {$\vec r$};
\end{tikzpicture}

Рассмотрим Z=0: $ae^{ik\sin\alpha x}=a^{i\Omega x}$, $\Omega=u\sin\alpha$- простая частота. $d=\frac{2\pi}{\Omega}$- простой период. 

\begin{tabular}{c|c}
Время  & Прост. \\ \hline
$e^{i\omega t}$ & $e^{i\Omega t}$ \\ 
$T=\frac{2\pi}{\omega}$ & $d=\frac{2\pi}{\Omega}$ \\
\end{tabular}

В плоскости t=0  мы имеем возмущение $a^{i\lambda\sin\alpha x}$, а что происходит при Z>0? А туда кашля плоская волна под углом $\alpha$, где $\sin\alpha=\frac{\Omega}{K}$, $\Omega$- простая частота, K- волновой вектор.

\begin{tikzpicture}[line cap=round,line join=round,>=triangle 45,x=1cm,y=1cm]
\clip(-2,-1.5) rectangle (8,3);
\draw [->,line width=1pt] (0,0) -- (2.2595001290053087,0);
\draw [->,line width=1pt] (0,0) -- (0,2.3100700521604565);
\draw (-0.3171136019442876,2.7833256353960962) node[anchor=north west] {x};
\draw (2.6626437739838167,-0.09126383314630822) node[anchor=north west] {z};
\draw (-0.4047535247657024,0.06648802793223836) node[anchor=north west] {0};
\draw (2,1.5) node[anchor=north west] {x,z};
\draw (0,-0.2) node[anchor=north west] {f(x)};
\end{tikzpicture}

f(x)- коллежская амплитуда поля в плоскости z=0.
Предположение, что у нас периодическое возмущение. Значит мы можем разложить его в ряд Фурье.

Z=0: f(x)=$\sum\limits_{-\infty}^{+\infty} f_ne^{i\Omega_n x}$, или $\int F|n|e^{i\Omega x}d\Omega$

Вывод: Каждому члену ряда  соответствует плоская волна, которая бежит в направлении $\sin\alpha_n=\frac{\Omega_n}{K}$, Z>0, $E=\sum F_ne^{i(\Omega_nx+\sqrt{K^2+\Omega^2_nz})}e^{-i\omega t}$

$e^{-i\omega t}$- формула передачи пространства( свободного).

\hspace{4cm}Линза как спектральный прибор

$\Omega=k\sin\alpha$ $\approx$ $k\tan\alpha=k \frac{\rho}{F}$, $\alpha<<\pi$

Плоские волны, простая линза, фокусируются на экране в Т.Р.
Линза- это спектральный прибор который существует пространстве преобразование Фурье.
Пространства, т.е координата преобразования  X.  

\begin{tikzpicture}[line cap=round,line join=round,>=triangle 45,x=1cm,y=1cm]
\clip(-2.9375906443694646,-2) rectangle (5.969090608750586,1.6);
\draw [rotate around={90:(0,0)},line width=2pt] (0,0) ellipse (1cm and 0.3cm);
\draw [line width=1pt] (-1.5,1.2)-- (-1.5,-1.2);
\draw [line width=1pt] (1.5,1.2)-- (1.5,-1.2);
\draw [line width=1pt] (0,1)-- (-1.5,0.3891430294360567);
\draw [line width=1pt] (-1.5,0)-- (-0.26754146751384356,0.4524201975123346);
\draw [line width=1pt] (-0.3,0)-- (-1.5,-0.5);
\draw [line width=1pt] (-0.27579581639162615,-0.393512214421137)-- (-1.5,-0.9342516705615564);
\draw [line width=1pt] (0,1)-- (1.5,0.7963413986660915);
\draw [line width=1pt] (-0.26754146751384356,0.4524201975123346)-- (0.26177625430777407,0.48845780757845497);
\draw [line width=1pt] (0.26177625430777407,0.48845780757845497)-- (1.5,0.7963413986660914);
\draw [line width=1pt] (-0.3,0)-- (0.3,0);
\draw [line width=1pt] (0.3,0)-- (1.5,0.7963413986660914);
\draw [line width=1pt] (1.5,0.7963413986660914)-- (0.14380470740013346,-0.4330844468938212);
\draw [line width=1pt] (0.14380470740013346,-0.4330844468938212)-- (-0.27579581639162615,-0.393512214421137);
\draw (-1.4043321694609838,1.4862294049962086) node[anchor=north west] {z=0};
\draw (-1.4646288510585084,-1.2357350785491792) node[anchor=north west] {f(x)};
\draw (1.6621847803559777,0.9263316473049105) node[anchor=north west] {P};
\draw (1.2917908791140413,1.503457028309787) node[anchor=north west] {-};
\draw [->,line width=1pt] (0.8955555429017373,-0.7619754374257731) -- (1.5,-0.7619754374257731);
\draw [->,line width=1pt] (0.6285273815412715,-0.7705892490825623) -- (0.1931251086398671,-0.7652355658578426);
\draw (0.723279309765953,-0.7705892490825623) node[anchor=north west] {F};
\draw [line width=1pt] (-2.2657133351399055,-0.021187634941901724)-- (2.385744959526272,0.030495234998833488);
\end{tikzpicture} 

$f(\Omega)$- $\int f(x)e^{-i\Omega x}dx$=$\int f(x)e^{-i\frac{K}{F}\rho x}dx$

В задней формальной плоскости возникает дифракционная Картина Фраунгофера. В задней фокусе плоскости возникает также и пространстве преобразование Фурье. Значит дифракционная картина Фраунгофера и есть пространстве Фурье преобразование.

$\Delta\Omega\Delta x \approx 2\pi$(*); $\Delta\omega r\approx 2\pi$(**) 

Если 2 формулы связаны Фурье преобразования, то их ширины связаны некоторым соотношением. 

(*) $\Rightarrow$  Если мы по координате x  имеем очень узкую область,  через которую проходит свет, тогда угловое распределение будет широкое.

\hspace{3cm} Формирование геометрического щобразования

\begin{tikzpicture}[line cap=round,line join=round,>=triangle 45,x=1cm,y=1cm]
\clip(-4.4,-2.4) rectangle (6,2);
\draw [rotate around={90:(-1,0)},line width=1pt] (-1,0) ellipse (1cm and 0.3cm);
\draw [rotate around={90:(1,0)},line width=1pt] (1,0) ellipse (1cm and 0.3cm);
\draw [line width=1pt] (-1,1)-- (1,1);
\draw [line width=1pt] (-1,-1)-- (1,-1);
\draw [line width=1pt] (-2,0)-- (-1,1);
\draw [line width=1pt] (-1,-1)-- (-2,0);
\draw [line width=1pt] (1,-1)-- (2,0);
\draw [line width=1pt] (1,1)-- (2,0);
\draw [line width=1pt,dash pattern=on 1pt off 1pt] (-2,-3.9226563322761985) -- (-2,3.1934379897940026);
\draw [line width=1pt,dash pattern=on 1pt off 1pt] (2,-3.9226563322761985) -- (2,3.1934379897940026);
\draw [->,line width=1pt] (-3.01,0.65) -- (-2.3483527359013663,0.6535741358475615);
\draw [->,line width=1pt] (-3,0) -- (-2.359541563451791,0);
\draw [->,line width=1pt] (-2.94,-0.67) -- (-2.359541563451791,-0.6667075151025701);
\draw [line width=1pt,dash pattern=on 1pt off 1pt] (0,1)-- (0,-1);
\draw [line width=1pt,dash pattern=on 1pt off 1pt] (-1.5651348073716274,1.2242043409192287)-- (-1.5651348073716274,-1.2261488926238124);
\draw [line width=1pt,dash pattern=on 1pt off 1pt] (2.66,1.37)-- (2.66,-1.26);
\draw (-0.32541271478455314,1.9) node[anchor=north west] {Фурье плоскость};
\draw (-2.6079335350712207,-1.6289466844391056) node[anchor=north west] {предел плоскости};
\draw [->,line width=1pt] (-0.42387439722829373,-1.3827924783297603) -- (0,-1.3827924783297603);
\draw [->,line width=1pt] (-0.77,-1.38) -- (-1.11,-1.38);
\draw (-0.7,-1.2485265477246608) node[anchor=north west] {F};
\draw (-1.8023379514406321,1.7612680633396225) node[anchor=north west] {f(x)};
\draw (2.2927729320148598,-1.29328185792636) node[anchor=north west] {f(-x)};
\draw (-2,-1.9) node[anchor=north west] {(1)};
\draw (1.4,-1.5) node[anchor=north west] {(2)};
\end{tikzpicture}

(1) и (2) выполняют обе прямое преобразование( двойная дифракционная). Первым применил Эрись Аббе в микрополе( работал в Care Zeiss).

Что происходит, если мы смещаем изображение?
Ответ: Когда мы наблюдаем глазам то, наш глаз фиксирует распределение интенсивности, т.е лводр. амплитуды в каждом месте. При сдвиге объема меняются фазовые соответствия.  

\definecolor{qqqqqq}{rgb}{0,0,0}
\begin{tikzpicture}[line cap=round,line join=round,>=triangle 45,x=1cm,y=1cm]
\clip(-4.8,-2.5) rectangle (11.309804759750513,2.5);
\draw [line width=1pt] (-3,0)-- (4,0);
\draw [rotate around={90:(-1,0)},line width=1pt] (-1,0) ellipse (1cm and 0.3cm);
\draw [line width=1pt] (2,1.5)-- (2,-1.5);
\draw [line width=1pt] (-2.489136051214823,0)-- (-1,1);
\draw [line width=1pt] (-2.489136051214823,0)-- (-1,-1);
\draw [line width=1pt] (-1,-1)-- (2,0);
\draw [line width=1pt] (2,0)-- (-1,1);
\draw [->,line width=1pt] (-3.7339047226535613,0.74) -- (-2.8312585506602446,0.7406946055066166);
\draw [->,line width=1pt] (-3.9139047226535606,0.2990068055247457) -- (-3.0939047226535608,0.2990068055247457);
\draw [->,line width=1pt] (-3.8956396600460006,-0.39971372597812227) -- (-3.039931944752544,-0.40297958342576307);
\draw [line width=1pt,dash pattern=on 1pt off 1pt] (-0.11276112289659412,0.7042537076321981)-- (-0.10740156721424854,-0.7024671890714163);
\draw [line width=1pt,dash pattern=on 1pt off 1pt] (-0.11276112289659412,0.7042537076321981)-- (-0.10740156721424854,-0.7024671890714163);
\draw (-1.2917073031558446,-1.2170063635421833) node[anchor=north west] {F};
\draw [shift={(-2.4891360512148233,0)},line width=1pt]  plot[domain=0:0.5913621898467183,variable=\t]({1*0.4891360512148233*cos(\t r)+0*0.4891360512148233*sin(\t r)},{0*0.4891360512148233*cos(\t r)+1*0.4891360512148233*sin(\t r)});
\draw [shift={(-2.4891360512148233,0)},line width=1pt]  plot[domain=0:0.5913621898467183,variable=\t]({1*0.6842449988908432*cos(\t r)+0*0.6842449988908432*sin(\t r)},{0*0.6842449988908432*cos(\t r)+1*0.6842449988908432*sin(\t r)});
\draw [line width=1pt,color=qqqqqq] (-1.7808433543706705,0.23430073919153985)-- (-1.7808433543706705,0.23430073919153985)-- (-1.761836548845925,0.3103279612905224)-- (-1.7238229377964338,0.3673483778647594)-- (-1.7048161322716882,0.46238240548848764)-- (-1.6668025212221969,0.5574164331122158)-- (-1.6668025212221969,0.7284776828349266)-- (-1.6668025212221969,0.8615253215081462)-- (-1.6858093267469425,0.9185457380823832)-- (-1.7238229377964338,0.9755661546566201)-- (-1.761836548845925,1.032586571230857)-- (-1.8188569654201618,1.0706001822803484)-- (-1.91389099304389,1.1086137933298397)-- (-1.9899182151428727,1.1276205988545853)-- (-2.0469386317171097,1.1656342099040766)-- (-2.1039590482913466,1.1846410154288223)-- (-2.1229658538160923,1.2416614320030592)-- (-2.1039590482913466,1.2986818485772962)-- (-2.0659454372418553,1.3557022651515331)-- (-2.0089250206676184,1.3937158762010244);
\draw (-1.8048910523239767,1.8430893259418644) node[anchor=north west] {u- Апертурный угол};
\end{tikzpicture}

\newpage
    Чтобы получить правильное соответствие между колониям Фурье плоскости, нужно обент поместил в переднюй фокальную плоскость.
     Аббе назвал то, что происходит в задней фокусной плоскости- первичным изображением, а в плоскости самого изображения- вторичным изображением.                         


\end{document}